\documentclass[11pt,letterpaper,english]{article}
\usepackage[width=7in,left=.75in,top=.75in,bottom=1in]{geometry}
%\usepackage[T1]{fontenc}
%\usepackage[latin1]{inputenc}
\usepackage{amsmath}
\usepackage{graphicx}
\usepackage{amssymb}
\usepackage{mathtools}
\usepackage{babel}
\usepackage{color}
\usepackage[colorlinks=true]{hyperref}
\usepackage{marginnote}
\usepackage{longtable}
\usepackage{url}
\usepackage[qualifier-mode=phrase,qualifier-phrase={\,}]{siunitx}
%\usepackage{wrapfig}

\DeclareSIUnit\Molar{\textsc{m}}
\DeclareSIUnit\muMolar{\milli\mol\per\meter\cubed}

\setlength\parskip{\medskipamount}
\setlength\parindent{0pt}

\newcommand{\qtext}[1]{\qquad\text{#1}}
\newcommand{\lxr}[1]{{\tt #1}}
\renewcommand{\ll}[1]{{\tt #1}}
\let\ds=\displaystyle

\let\norm=\|
\def\|#1|{\operatorname{#1}}

\def\hasSi{\|hasSi|_j}
\def\hasPIC{\|hasPIC|_j}
\def\diazo{\|diazo|_j}
\def\useNHiv{\|useNH4|_j}
\def\useNOii{\|useNO2|_j}
\def\useNOiii{\|useNO3|_j}
\def\combNO{\|combNO|_j}
\def\tempMort{e^{\|mort|}_j}
\def\tempMortTWO{e^{\|mort2|}_j}
\def\apSUBtype{\|ap\_type|}
\def\Xmin{c^{\min}_j}
\def\amminhib{\sigma_1}
\def\acclimtimescl{\tau^{\|acclim|}}
\def\PCmax{P^{\|max|}_{\|C|,j}}
\def\mort{m^{(1)}_j}
\def\mortTWO{m^{(2)}_j}
\def\mortTempFuncMin{f^{\|mort|\min}_j}
\def\mortTWOTempFuncMin{f^{\|mort2|\min}_j}
\def\ExportFrac{f^{\exp}_j}
\def\ExportFracMort{f^{\exp\|mort|}_j}
\def\ExportFracMortTWO{f^{\exp\|mort2|}_j}
\def\phytoTempCoeff{c_j}
\def\phytoTempExpONE{e_{1j}}
\def\phytoTempExpTWO{e_{2j}}
\def\phytoTempOptimum{T^{\|opt|}_j}
\def\phytoDecayPower{p_j}
\def\RSUBNC{R^{\N:\C}_j}
\def\RSUBPC{R^{\P:\C}_j}
\def\RSUBSiC{R^{\Si:\C}_j}
\def\RSUBFeC{R^{\Fe:\C}_j}
\def\RSUBChlC{R^{\chlc}_j}
\def\RSUBPICPOC{R^{\PICPOC}_j}
\def\wsink{w^{\|sink|}_j}
\def\wswim{w^{\|swim|}_j}
\def\ksatNHiv{k^{\NHiv}_j}
\def\ksatNOii{k^{\NOii}_j}
\def\ksatNOiii{k^{\NOiii}_j}
\def\ksatPOiv{k^{\POiv}_j}
\def\ksatSiOii{k^{\SiOii}_j}
\def\ksatFeT{k^{\Fe}_j}
\def\ksatPAR{k^{\|sat|}_{\|PAR|}}
\def\kinhPAR{k^{\|inh|}_{\|PAR|}}
\def\inhibcoefSUBgeid{c^{\|inhib|}_j}
\def\mQyield{\Phi_j}
\def\chlTWOcmax{\chlc^{\max}_j}
\def\grazemax{g^{\max}_z}
\def\kgrazesat{k^{\|graz|}_z}
%\def\palat{p_{j,z}}
\def\asseff{a_{j,z}}
\def\ExportFracPreyPred{f^{\|exp|\|graz|}_{j,z}}
\def\respiration{r^{\|resp|}_j}
\def\Qnmax{Q^{\N\max}_j}
\def\Qnmin{Q^{\N\min}_j}
\def\Qpmax{Q^{\P\max}_j}
\def\Qpmin{Q^{\P\min}_j}
\def\Qsimax{Q^{\Si\max}_j}
\def\Qsimin{Q^{\Si\min}_j}
\def\Qfemax{Q^{\Fe\max}_j}
\def\Qfemin{Q^{\Fe\min}_j}
\def\VmaxSUBNHiv{V^{\NHiv\max}_j}
\def\VmaxSUBNOii{V^{\NOii\max}_j}
\def\VmaxSUBNOiii{V^{\NOiii\max}_j}
\def\VmaxSUBN{V^{\N\max}_j}
\def\VmaxSUBPOiv{V^{\POiv\max}_j}
\def\VmaxSUBSiOii{V^{\SiOii\max}_j}
\def\VmaxSUBFeT{V^{\Fe\max}_j}
\def\kexcC{\kappa^{\|exc|}_{\C j}}
\def\kexcN{\kappa^{\|exc|}_{\N j}}
\def\kexcP{\kappa^{\|exc|}_{\P j}}
\def\kexcSi{\kappa^{\|exc|}_{\Si j}}
\def\kexcFe{\kappa^{\|exc|}_{\Fe j}}
\def\aphySUBchl{a^{\chl}_{\|phy| {j,l}}}
\def\aphySUBchlSUBps{a^{\chl}_{\|ps| {j,l}}}
\def\bphySUBchl{b^{\chl}_{\|phy| {j,l}}}
\def\bbphySUBchl{b^{\chl}_{\|b|\|phy| {j,l}}}

\newcommand{\ee}{\mathrm{e}}
\newcommand{\DDt}{\operatorname{D}_t}

\DeclareMathOperator{\DIC}{DIC}
\DeclareMathOperator{\PO}{PO}
\DeclareMathOperator{\Fe}{Fe}
\DeclareMathOperator{\FeL}{FeL}
\DeclareMathOperator{\Si}{Si}
\DeclareMathOperator{\SiO}{SiO}
\DeclareMathOperator{\NH}{NH}
\DeclareMathOperator{\NO}{NO}
\DeclareMathOperator{\Alk}{Alk}
\DeclareMathOperator{\DOM}{DOM}
\DeclareMathOperator{\DOC}{DOC}
\DeclareMathOperator{\DOP}{DOP}
\DeclareMathOperator{\DON}{DON}
\DeclareMathOperator{\DOFe}{DOFe}
\DeclareMathOperator{\POM}{POM}
\DeclareMathOperator{\POP}{POP}
\DeclareMathOperator{\PON}{PON}
\DeclareMathOperator{\POFe}{POFe}
\DeclareMathOperator{\POSi}{POSi}
\DeclareMathOperator{\PIC}{PIC}
\DeclareMathOperator{\POC}{POC}
\DeclareMathOperator{\CDOM}{CDOM}
\DeclareMathOperator{\chl}{chl}
\DeclareMathOperator{\chlc}{chl\text{:}c}
\DeclareMathOperator{\chln}{chl\text{:}n}
\DeclareMathOperator{\POiv}{PO4}
\DeclareMathOperator{\NHiv}{NH4}
\DeclareMathOperator{\NOii}{NO2}
\DeclareMathOperator{\NOiii}{NO3}
\DeclareMathOperator{\SiOii}{SiO2}

\DeclareMathOperator{\QFe}{\mathit{Q}Fe}

\newcommand{\N}{\mathrm{N}}
\newcommand{\F}{\mathrm{F}}
\newcommand{\C}{\mathrm{C}}
\renewcommand{\P}{\mathrm{P}}
\renewcommand{\O}{\mathrm{O}}
\newcommand{\fe}{\mathrm{fe}}
\let\unit=\si
\renewcommand{\si}{\mathrm{si}}

\newcommand{\m}{\operatorname{m}}

\newcommand{\total}{{\mathrm{T}}}
\newcommand{\FeT}{\Fe_\total}
\newcommand{\Ltot}{L_\total}

\newcommand{\sed}{{\text{sed}}}
\newcommand{\stab}{\text{stab}}
\newcommand{\Lstab}{\beta_{\stab}}
\newcommand{\tot}{\text{tot}}

\newcommand{\PICPOC}{{\text{PIC:POC}}}
\newcommand{\diss}{{\text{diss}}}
\newcommand{\resp}{{\text{resp}}}
\newcommand{\graz}{{\text{graz}}}
\newcommand{\scav}{{\text{scav}}}
\newcommand{\free}{{\text{free}}}
\newcommand{\sink}{{\text{sink}}}
\newcommand{\reg}{{\text{reg}}}
\newcommand{\phy}{{\text{phy}}}
\newcommand{\zoo}{{\text{zoo}}}
\newcommand{\up}{{\text{up}}}
\newcommand{\inhib}{{\text{inhib}}}
\newcommand{\acclim}{{\text{acclim}}}
\newcommand{\nit}{{\text{nit}}}
\newcommand{\oxi}{{\text{oxi}}}
\newcommand{\remin}{{\text{remin}}}
\newcommand{\crit}{{\text{crit}}}
\newcommand{\export}{{\text{exp}}}
\renewcommand{\day}{{\text{day}}}
%\newcommand{\mort}{{\text{mort}}}

\newcommand{\X}{c}
\newcommand{\palat}{p}
\newcommand{\eps}{{\mathtt{EPS}}}


\begin{document}

\section{GUD ecosystem model}

Inorganic nutrients:
\begin{align*}
  \DDt \DIC   &= -\sum_j U^{\DIC}_j \cdot (1 + R^{\PICPOC}_j)
                 + r_{\DOC} \DOC + (r_{\POC} \POC)
                 + \kappa^\diss_\C \PIC \\
  \DDt \PO_4  &= -\sum_j U^{\POiv}_j + r_{\DOP} \DOP + (r_{\POP} \POP) \\
  \DDt \NH_4  &= -\sum_j U^{\NHiv}_j + r_{\DON} \DON + (r_{\PON} \PON) - P_{\NOii}
                                     - D_{\NHiv} \\
  \DDt \NO_2  &= -\sum_j U^{\NOii}_j + P_{\NOii} - P_{\NOiii} \\
  \DDt \NO_3  &= -\sum_j U^{\NOiii}_j + P_{\NOiii} - D_{\NOiii} \\
  \DDt \SiO_2 &= -\sum_j U^{\SiOii}_j + r_{\POSi} \POSi \\
  \DDt \FeT   &= -\sum_j U^{\Fe}_j + r_{\DOFe} \DOFe + (r_{\POFe} \POFe) + S_{\Fe} \\[-1ex]
\noalign{Plankton:}
  \DDt \X_j   &= U^{\DIC}_j - M_j - R^\C_j - G_j + g_j^\C \\
\noalign{with P quota:}
  \DDt p_j    &= U^{\POiv}_j - M_j Q^\P_j - G_j Q^\P_j + g_j^\P \\
\noalign{with N quota:}
  \DDt n_j    &= U^{\N}_j - M_j Q^\N_j - G_j Q^\N_j + g_j^\N \\
\noalign{with Si quota:}
  \DDt \si_j  &= U^{\SiOii}_j - M_j Q^{\Si}_j - G_j Q^{\Si}_j \\
\noalign{with Fe quota:}
  \DDt \fe_j  &= U^{\Fe}_j - M_j Q^{\Fe}_j - G_j Q^{\Fe}_j + g_j^{\Fe} \\
\noalign{with Chl quota:}
  \DDt \chl_j &= S^{\chl}_j - M_j \cdot \chlc_j - G_j \cdot \chlc_j
\end{align*}
With \verb|GUD_ALLOW_CDOM|, all particulate remineralization terms (in parenthesis)
except Si are absent.  Terms: $U$ -- uptake; $P$ -- nitrification;
$D$ -- denitrification; $R$ -- respiration; $G$, $g$ -- grazing;
$S$ -- synthesis/other sources.



\subsection{Geider growth}

\[
  U^{\DIC}_j = P^\C_j \X_j
      \underbrace{{}-{\tt synthcost}\cdot U^\N_j}_{\mathclap{
                      \text{only with both N and Chl quotas}}}
\]

\[
  P^\C_j = P^{\C\m}_j \left(
    1 - \exp\left\{ -\frac{\gamma^{\QFe}_j
                     \langle\alpha I\rangle_j
                     \chlc_j }{ P^{\C\m}_j }
            \right\}
  \right) \gamma^\inhib_j
  \cdot{\tt ngrow}
  \qquad\text{if } I>I_{\min} \text{ and } P^{\C\m}_j > 0 \text{, else } 0
\]
\[
  \langle\alpha I\rangle_j = \sum_{l=1}^{\tt nlam} \alpha^{\chl}_{j,l} I_l
\]
\[
  P^{\C\m}_j = P_{\C,j}^{\max} \gamma^{\|nut|}_j f^\phy_j(T) \gamma_{\|pCO2|}
\]
(${\tt ngrow}=\gamma_{\|pCO2|}=1$).

Without dynamic Chl quota, we set\footnote{Notation:
$[A]_a^b\equiv\max(a,\min(b,A))$}
\[
  \chlc_j = \chlc_j^{\|acclim|}
\]
\[
  \chlc_j^{\|acclim|} = \left[
      \frac{\chlc^{\max}{_j}}
           {1 + \chlc^{\max}_j\langle\alpha I\rangle_j/(2P^{\C\m}_j)}
    \right]_{\chlc^{\min}_j}^{\chlc^{\max}_j}
  \qquad \text{if } P^{\C\m}_j \ne 0 \text{, else } \chlc^{\min}_j .
\]

Photo inhibition:
\[
  \gamma^\inhib_j = \begin{cases}
    c^\inhib_j \cdot \verb|EkoverE| & \text{if } \verb|EkoverE| \le 1 \\
    1                               & \text{otherwise}
  \end{cases}
\]
where
\[
  \verb|EkoverE| = \frac{P^{\C\m}_j/(\chlc_j\cdot\overline{\alpha}_j)}
                   {\langle\alpha I\rangle_j/\overline{\alpha}_j}
\]
and
\[
  \overline{\alpha}_j = \sum_l \Delta\lambda_l \alpha^{\chl}_{j,l} \Big/
                        \sum_l \Delta\lambda_l
  \;.
\]



\subsection{Nutrient uptake and limitation}

\[
  \gamma^{\|nut|}_j = \min(\gamma^\P_j, \gamma^\N_j, \gamma^{\Si}_j,
                           \gamma^{\Fe}_j)
\]

\paragraph{Without P quota:} Monod limitation
\[
  \gamma^\P_j = \frac{\mathrm{PO}_4}{\mathrm{PO}_4 + k^{\POiv}_j}
\]
\[
  U^\P_j = R^{\P:\C}_j P^\C_j \X_j
\]
\paragraph{With P quota:} normalized Droop limitation
\[
  \gamma^\P_j = \left[ \frac{1 - Q^{\P\min}_j/Q^{\P}_j}
                            {1 - Q^{\P\min}_j/Q^{\P\max}_j}
                \right]_0^1
\]
\[
  U^\P_j = V^{\P\max}_j \frac{\mathrm{PO}_4}{\mathrm{PO}_4 + k^{\POiv}_j}
                \left[ \frac{Q^{\P\max}_j - Q^{\P}_j}
                            {Q^{\P\max}_j - Q^{\P\min}_j}
                \right]_0^1
           f^\up_j(T) \cdot \X_j
\]

\paragraph{Si:} Diatoms ($\|hasSi|=1$) have linear limitation when using a Si quota,
\[
  \gamma^{\Si}_j = \left[ \frac{Q^{\Si}_j - Q^{\Si\min}_j}
                           {Q^{\Si\max}_j - Q^{\Si\min}_j}
                   \right]_0^1
\]
Otherwise Si is analogous to P.


\paragraph{Without N quota:}

-- diazotroph: no limitation, no consumption
\[
  \gamma^\N_j = 1
\]
\[
  U^{\NHiv}_j = U^{\NOii}_j = U^{\NOiii}_j = 0
\]

-- not diazotroph: modified Monod limitation
\[
  \gamma^\N_j = \left[ \gamma^{\NHiv}_j + \gamma^{\NOii}_j + \gamma^{\NOiii}_j \right]_0^1
\]
\[
  \gamma^{\NHiv}_j = \verb|useNH4| \cdot \frac{\NH_4}{\NH_4 + k^{\NHiv}_j}
\]
\[
  \gamma^{\NOii}_j = \verb|useNO2| \cdot
  \ee^{-\sigma_1 \NH_4} \cdot
  \begin{cases}
    \dfrac{\NO_2}{\NO_2 + k^{\NOii}_j} & \text{if not }\verb|combNO| \\[3ex]
    \dfrac{\NO_2}{\NO_2 + \NO_3 + k^{\NOiii}_j} & \text{if }\verb|combNO|
  \end{cases}
\]
\[
  \gamma^{\NOiii}_j = \verb|useNO3| \cdot
  \ee^{-\sigma_1 \NH_4} \cdot
  \begin{cases}
    \dfrac{\NO_3}{\NO_3 + k^{\NOiii}_j} & \text{if not }\verb|combNO|_j \\[3ex]
    \dfrac{\NO_3}{\NO_2 + \NO_3 + k^{\NOiii}_j} & \text{if }\verb|combNO|_j
  \end{cases}
\]
\begin{align}
  U^{\NHiv}_j &= \frac{\gamma^{\NHiv}_j}
                 {\gamma^{\NHiv}_j + \gamma^{\NOii}_j + \gamma^{\NOiii}_j + \eps}
                 R^{\N:\C}_j P^\C_j \X_j
  \label{eq:UNH4} \\
  U^{\NOii}_j &= \frac{\gamma^{\NOii}_j}
                 {\gamma^{\NHiv}_j + \gamma^{\NOii}_j + \gamma^{\NOiii}_j + \eps}
                 R^{\N:\C}_j P^\C_j \X_j
  \label{eq:UNO2} \\
  U^{\NOiii}_j &= \frac{\gamma^{\NOiii}_j}
                  {\gamma^{\NHiv}_j + \gamma^{\NOii}_j + \gamma^{\NOiii}_j + \eps}
                  R^{\N:\C}_j P^\C_j \X_j
  \label{eq:UNO3}
\end{align}


\paragraph{With N quota:} linear limitation
\[
  \gamma^\N_j = \left[ \frac{Q^{\N}_j - Q^{\N\min}_j}
                        {Q^{\N\max}_j - Q^{\N\min}_j}
                \right]_0^1
\]
\begin{align*}
  U^{\NHiv}_j &= V^{\NHiv\max}_j
                 \frac{\NH_4}{\NH_4 + k^{\NHiv}_j}
                 \left[ \frac{Q^{\N\max}_j - Q^{\N}_j}
                             {Q^{\N\max}_j - Q^{\N\min}_j}
                 \right]_0^1
                 f^\up_j(T) \cdot \X_j \\
  U^{\NOii}_j &= V^{\NOii\max}_j \cdot
                 \ee^{-\sigma_1 \NH_4} \cdot
                 \frac{\NO_2}{\NO_2 + k^{\NOii}_j}
                 \left[ \frac{Q^{\N\max}_j - Q^{\N}_j}
                             {Q^{\N\max}_j - Q^{\N\min}_j}
                 \right]_0^1
                 f^\up_j(T) \cdot \X_j \\
  U^{\NOiii}_j &= V^{\NOiii\max}_j \cdot
                  \ee^{-\sigma_1 \NH_4} \cdot
                  \frac{\NO_3}{\NO_3 + k^{\NOiii}_j}
                  \left[ \frac{Q^{\N\max}_j - Q^{\N}_j}
                              {Q^{\N\max}_j - Q^{\N\min}_j}
                  \right]_0^1
                  f^\up_j(T) \cdot \X_j
                  \cdot \gamma^{\QFe}_j
\end{align*}

-- diazotroph: consume what is available, fix what is missing (up to $V^{\N\max}_j$),
\[
  U^{\N}_j = \max\biggl( U^{\NHiv}_j + U^{\NOii}_j + U^{\NOiii}_j,\;
                 V^{\N\max}_j
                 \left[ \frac{Q^{\N\max}_j - Q^{\N}_j}
                             {Q^{\N\max}_j - Q^{\N\min}_j}
                 \right]_0^1
                 f^\up_j(T) \cdot \X_j
             \biggr) \\
\]
%with $V^{\N\max}_j \ge V^{\NHiv\max}_j + V^{\NOii\max}_j + V^{\NOiii\max}_j$.
Rate of nitrogen fixation is
\[
  U^{\N}_j - U^{\NHiv}_j - U^{\NOii}_j - U^{\NOiii}_j
\]
\textbf{DON excretion?}


-- not diazotroph:
\[
  U^{\N}_j = U^{\NHiv}_j + U^{\NOii}_j + U^{\NOiii}_j
\]


\paragraph{Without Fe quota:}
\[
  \gamma^{\Fe}_j = \frac{\FeT}{\FeT + k^{\Fe}_j}
\]
\[
  \gamma^{\QFe}_j = 1
\]
\[
  U^{\Fe}_j = R^{\Fe:\C}_j P^\C_j \X_j
\]

\paragraph{With Fe quota,} iron limitation (normalized Droop) moves to
photosynthesis and nitrate uptake,
\[
  \gamma^{\Fe}_j = 1
\]
\[
  \gamma^{\QFe}_j = \left[ \frac{1 - Q^{\Fe\min}_j/Q^{\Fe}_j}
                                {1 - Q^{\Fe\min}_j/Q^{\Fe\max}_j}
                    \right]_0^1
\]
\[
  U^{\Fe}_j = V^{\Fe\max}_j \frac{\FeT}{\FeT + k^{\Fe}_j}
                \left[ \frac{Q^{\Fe\max}_j - Q^{\Fe}_j}
                            {Q^{\Fe\max}_j - Q^{\Fe\min}_j}
                \right]_0^1
           f^\up_j(T) \cdot \X_j
\]



\subsection{Chlorophyll synthesis}

\paragraph{With N quota:}
\[
  S^{\chl}_j = \rho^{\chl}_j U^\N_j
\]
where
\[
  \rho^{\chl}_j = \chln^{\max}_j \frac{P^\C_j}
                             {\langle\alpha I\rangle_j \cdot \chlc_j}
  \qtext{if } \langle\alpha I\rangle_j \cdot \chlc_j > 0
  \text{, else } \chln^{\max}_j \;.
\]
%Also, DIC uptake is reduced by the cost of synthesis, $\verb|synthcost|\cdot U^\N_j$.

\paragraph{Without N quota:}~

-- with \verb|GUD_GEIDER_RHO_SYNTH|:
\[
  S^{\chl}_j = \rho^{\chl}_j \cdot P^\C_j \X_j
           + \tau^{\|acclim|}_j (\chlc_j^{\|acclim|} - \chlc_j) \X_j
\]
where
\[
  \rho^{\chl}_j = \chlc^{\max}_j \frac{P^\C_j}
                                {\langle\alpha I\rangle_j \chlc^\acclim_j}
  \qtext{if } \langle\alpha I\rangle_j > 0 \text{ and } \chlc^\acclim_j > 0
  \text{, else } \mathbf{0} \;.
\]

-- else
\[
  S^{\chl}_j = \chlc_j^{\|acclim|} \cdot P^\C_j \X_j
             + \tau^{\|acclim|}_j (\chlc_j^{\|acclim|} - \chlc_j) \X_j
\]

\paragraph{Without Chl quota,} current chl, i.e., $\X_j\cdot\chlc_j$,
is stored for the next time step.



\subsection{Organic matter pools}

\begin{align*}
  \DDt \DOC  &= \sum_j M_j^{\DOM} \;\;\;\;\;+ g^{\DOC}  - r_{\DOC} \DOC
\\
  \DDt \DOP  &= \sum_j M_j^{\DOM} Q_j^{\P}  + g^{\DOP}  - r_{\DOP} \DOP
\\
  \DDt \DON  &= \sum_j M_j^{\DOM} Q_j^{\N}  + g^{\DON}  - r_{\DON} \DON
\\
  \DDt \DOFe &= \sum_j M_j^{\DOM} Q_j^{\Fe} + g^{\DOFe} - r_{\DOFe} \DOFe
\\
  \DDt \PIC  &= \sum_j M_j^{\DOM} R_j^{PICPOC} + g^{\PIC} - \kappa^\diss_\C \PIC
\\
  \DDt \POC  &= \sum_j M_j^{\POM} \;\;\;\;\;+ g^{\POC}  - r_{\POC} \POC
\\
  \DDt \POP  &= \sum_j M_j^{\POM} Q_j^{\P}  + g^{\POP}  - r_{\POP} \POP
\\
  \DDt \PON  &= \sum_j M_j^{\POM} Q_j^{\N}  + g^{\PON}  - r_{\PON} \PON
\\
  \DDt \POSi &= \sum_j M_j^{\POM} Q_j^{\Si} + g^{\POSi} - r_{\POSi} \POSi
\\
  \DDt \POFe &= \sum_j M_j^{\POM} Q_j^{\Fe} + g^{\POFe} - r_{\POFe} \POFe
\end{align*}



\subsection{Remineralization}

Remineralization rates are temperature dependent:
\[
  r_{*O*} = f^\remin(T) \kappa_{*O*}
\]

nitrogen chemistry:
\begin{align*}
  P_{\NOii}  &= \kappa^{\|nit|}_{\rm a} \NH_4 \gamma^\nit \\
  P_{\NOiii} &= \kappa^{\|nit|}_{\rm b} \NO_2 \gamma^\nit
\end{align*}
where
\[
  \gamma^\nit = \max(0, 1-I/I_\oxi)
  \qtext{if } I_\oxi > 0 \text{, else } 1 \;.
\]



\subsection{Dynamic CDOM ({\tt \#define GUD\_ALLOW\_CDOM})}

Remineralization of particulate organic matter to inorganic matter is replaced by
\begin{align*}
  \DDt \CDOM &= f_{\CDOM} (r_{\POP} \POP + g^{\DOP}) - D_{\CDOM}
  \\
  \DDt \DOP  &= \dots + D_{\CDOM} - f_{\CDOM} (r_{\POP} \POP + g^{\DOP}) \\
  \DDt \DOC  &= \dots + R^{\C:\P}_{\CDOM} (D_{\CDOM} - f_{\CDOM} (r_{\POP} \POP + g^{\DOP})) \\
  \DDt \DON  &= \dots + R^{\N:\P}_{\CDOM} (D_{\CDOM} - f_{\CDOM} (r_{\POP} \POP + g^{\DOP})) \\
  \DDt \DOFe &= \dots + R^{\Fe:\P}_{\CDOM}(D_{\CDOM} - f_{\CDOM} (r_{\POP} \POP + g^{\DOP}))
\end{align*}
where
\begin{align*}
  D_{\CDOM} &= \left( r^{\|degrad|}_{\CDOM}
                  + r^{\|bleach|}_{\CDOM} \min(1, I/I_{\CDOM}) \right)
             f^\remin(T) \cdot \CDOM
\end{align*}
($\CDOM$ is still in P units.)



\subsection{Denitrification ({\tt \#define GUD\_ALLOW\_DENIT})}

When $\O_2<\O_2^\crit$, denitrification occurs:
\begin{align*}
  D_{\NHiv}  &= r_{\DON} \DON + (r_{\PON} \PON)
  \qtext{if } \O_2<\O_2^\crit \text{, else } 0 \\
  D_{\NOiii} &= {\tt denit\_NO3} \cdot \bigl(r_{\DOP} \DOP + (r_{\POP} \POP) \bigr)
  \qtext{if } \O_2<\O_2^\crit \text{, else } 0
\end{align*}
(the particulate organic terms are not present with \verb|GUD_ALLOW_CDOM|).

When $\O_2<\O_2^\crit$ and $\NO_3<\NO_3^\crit$ (and \verb|GUD_ALLOW_DENIT|),
all remineralization, denitrification and CDOM degradation (except
bleaching) stops.



\subsection{Carbon chemistry}

\begin{align*}
  \DDt \Alk &= -\biggl( P_{\NOiii} - \sum_j U^{\NOiii}_j \biggr)
    - 2\biggl( \sum_j P^\C_j \X_j R^\PICPOC_j - \kappa^\diss_\C \PIC \biggr)
    + D_{\NOiii}
\\
  \DDt \O_2 &= R_{\O:\P} \biggl( \sum_j U^{\POiv}_j
           - r_{\DOP} \DOP - (r_{\POP} \POP)
         \biggr)
\end{align*}
The POP term is not present with \verb|GUD_ALLOW_CDOM|.

\dots surface forcing \dots



\subsection{Iron chemistry}

\[
  S_{\Fe} = \delta_{k,1} \frac{\alpha_{\Fe}}{\Delta r_\F h_\C} F_{\Fe}
          + \delta_{k\le{\tt kMaxFeSed}\text{ bottom}}
            \frac{1}{\Delta r_\F h_\C} F_{\Fe}^\sed
          - r_\scav \Fe'
\]
where
\[
  F_{\Fe}^\sed = \begin{cases}
    {\tt fesedflux\_pcm}\cdot w^\P_\sink R^{\C:\P}_\sed \POP
    & \text{if \tt GUD\_IRON\_SED\_SOURCE\_VARIABLE}, \\
    {\tt fesedflux}
    & \text{else.}
    \end{cases}
\]
and
\[
  r_\scav = \begin{cases}
    r_\scav I_\scav \POC^{e_\scav}
    & \text{if \tt GUD\_PART\_SCAV}, \\
    {\tt scav}
    & \text{else.}
    \end{cases}
\]
If \verb|GUD_PART_SCAV_POP| is defined, $\POC$ is replaced by
$\POP\!/R^{\POP:\POC}_\scav$.

%The free iron concentration $\Fe'$ is determined as detailed in App.~\ref{app:fe}.
%\subsection{Fe chemistry}\label{app:fe}

The rate of iron scavenging depends on the amount of free dissolved iron which
is determined following \cite{Parekh2004}, \cite{Dutkiewicz2005}.
Free dissolved iron, $\Fe'$, is assumed to be in equilibrium with dissolved iron
bound to ligands via
\[
  \Fe' + L' \rightleftharpoons \FeL
\]
i.e.,
\[
  \frac{[\FeL]}{[\Fe'][L']} = \Lstab
  \;,
\]
given $\FeL+\Fe'=\FeT$ and $\FeL+L'=\Ltot$.  The solution is
\begin{align*}
 \begin{split}
  L' &= \frac{ \Lstab (\Ltot - \FeT) - 1
             +\sqrt{(1 - \Lstab (\Ltot - \FeT))^2 + 4 \Lstab \Ltot}}
          {2 \Lstab}
    \\
  \FeL &= \Ltot - L' \\
  \Fe' &= \FeT - \FeL
%       &= \tfrac12 \left(
%          \FeT - \Ltot - 1/\Lstab
%             + \sqrt{(1/\Lstab - (\Ltot - \FeT))^2 + 4 \Ltot/\Lstab}
%         \right)
  \;.
 \end{split}
  \\
\end{align*}
If \lxr{GUD\_MINFE} is defined, $\Fe'$ will be constrained to be no more than $\Fe'_{\max}$,
and $\FeT$ adjusted accordingly, assuming that excess free iron is scavenged away.
This is done before and after each biogeochemical subtimestep as the above reaction
is very fast.

%By the same reasoning, one could actually solve for equilibrium under the condition
%that $\Fe'$ stay below $\Fe'_{\max}$, thus imposing a limit directly on $\FeT$.

%The effective scavenging rate of total dissolved iron, $c_{\scav}\Fe'/\FeT$, is shown
%in Fig.\ \ref{fig:scavenging}.
%
%\begin{figure}
%\centering
%\includegraphics[width=.7\textwidth]{scavenging.pdf}
%\caption{Dissolved-iron scavenging rate and free iron concentration as
%a function of total dissolved iron.}
%\label{fig:scavenging}
%\end{figure}

\begin{tabular}{@{}l@{\qquad}l@{${}={}$}l}
\ll{ligand\_tot}  & $\Ltot=\SI{e-3}{\muMolar}$
                  & total ligand (\unit{\muMolar})  \\
\ll{ligand\_stab} & $\Lstab=1/(\SI{5E-6}{\muMolar})$
                  & ligand stablity rate ratio (\unit{\meter\cubed\per\milli\mol}) \\
\ll{freefemax}    & $\Fe'_{\max}=\SI{0.4E-3}{\muMolar}$
                  & maximal concentration of free iron (\unit{\milli\mol\per\meter\cubed})
\end{tabular}






\subsection{Mortality and Respiration}

Respiration and mortality stop at $\X_j^{\min}$
(maybe should not use $\X_j^{\min}$ for respiration?)
\[
  R^\C_j = r^\resp_j f^\remin(T) (\X_j - \X_j^{\min})
\]
\[
  M_j = m^{(1)}_j {f^{\|mort|}(T)}^{\tempMort} (\X_j - \X_j^{\min})
      + m^{(2)}_j {f^{\|mort2|}(T)}^{\tempMortTWO} (\X_j - \X_j^{\min})^2
\]
The released matter splits into dissolved and particulate organic pools,
\[
  M^{\DOM}_j = (1 - f_j^{\exp\|mort|}) m^{(1)}_j {f^{\|mort|}(T)}^{\tempMort} (\X_j - \X_j^{\min})
            + (1 - f_j^{\exp\|mort2|}) m^{(2)}_j {f^{\|mort2|}(T)}^{\tempMortTWO} (\X_j - \X_j^{\min})^2
\]
\[
  M^{\POM}_j = f_j^{\exp\|mort|} m^{(1)}_j {f^{\|mort|}(T)}^{\tempMort} (\X_j - \X_j^{\min})
            + f_j^{\exp\|mort2|} m^{(2)}_j {f^{\|mort2|}(T)}^{\tempMortTWO} (\X_j - \X_j^{\min})^2
\]



\subsection{Exudation}

If {\verb|GUD_ALLOW_EXUDE| is defined, an additional per-element loss term
is introduced, e.g.,
\[
  \DDt p_j = \dots - E^\P_j
\]
with
\[
  E^\P_j = \kappa^{\|exc|}_{\P\,j} (\X_j - \X_j^{\min}) Q^\P_j
\]
and fed into organic matter pools, e.g.,
\[
  \DDt \DOP = \dots + \sum_j (1 - f^{\exp}_j) E^\P_j
\]
and
\[
  \DDt \POP = \dots + \sum_j f^{\exp}_j E^\P_j
\]
(Maybe should NOT use $\X_j^{\min}$ in this case, or have a separate one for each quota?)



\subsection{Grazing}

Grazing loss of plankton $j$:
\[
  G_j = \sum_{z\in\|pred|} G_{j,z}
\]
where
\[
  G_{j,z} = g^{\max}_z
       \frac{(\palat_{j,z} \X_j)^s}{A_z}
       \frac{p_z^h}{p_z^h + {k^\graz_z}^h}
       (1 - \ee^{-i_\graz p_z})
       f^\graz_z(T)
       \X_z
\]
with
\[
  A_z = \biggl[ \sum_j (\palat_{j,z} \X_j)^s \biggr]_{\ge\tt phygrazmin}
\]
\[
  p_z = \biggl[ \sum_j \palat_{j,z} \X_j - {\tt phygrazmin} \biggr]_{\ge 0}
\]
$s$ is 1 for non-switching and 2 for switching grazers (\verb|#define GUD_GRAZING_SWITCH|).
The exponent $h$ defaults to 1.

\textbf{Note:} For non-switching grazers ($s=1$), Ben has an additional factor
\[
  \frac{S^{[j]}_z}{S^\phy_z + S^\zoo_z}
\]
in $G_{j,z}$ where
\begin{align*}
  S^\phy_z &= \sum_{j\in\phy} \palat_{j,z} \X_j
\\
  S^\zoo_z &= \sum_{j\in\zoo} \palat_{j,z} \X_j
\end{align*}
and $S^{[j]}_z$ is the sum for the class plankton $j$ belongs to.
\textbf{This is not implemented yet!}

Gains from grazing:\\
-- plankton species $z$:
\begin{align*}
  g^{\C}_z &= \sum_j G_{j,z} a_{j,z} \reg^{Q\C}_z
\\
  g^{\P}_z &= \sum_j G_{j,z} a_{j,z} \reg^{Q\P}_z Q^\P_j
  \qquad\text{if GUD\_ALLOW\_PQUOTA}
\\
  &\dots\displaybreak[0]
\\
\intertext{-- organic matter pools:}
  g^{\DOC} &= \sum_{j,z} G_{j,z} (1 - a_{j,z} \reg^{Q\C}_z) (1 - f^{\exp\graz}_z)
\\
  g^{\DOP} &= \sum_{j,z} \begin{cases}
    G_{j,z} (1 - a_{j,z} \reg^{Q\P}_z) (1 - f^{\exp\graz}_z) Q^\P_j
    &\text{if GUD\_ALLOW\_PQUOTA} \\
    G_{j,z} (R^{\P:\C}_j - a_{j,z} R^{\P:\C}_z) (1 - f^{\exp\graz}_z)
    &\text{else}
  \end{cases}
\\
  &\dots\displaybreak[0]
\\
  g^{\POC} &= \sum_{j,z} G_{j,z} (1 - a_{j,z} \reg^{Q\C}_z) f^{\exp\graz}_z
\\
  g^{\POP} &= \sum_{j,z} \begin{cases}
    G_{j,z} (1 - a_{j,z} \reg^{Q\P}_z) f^{\exp\graz}_z Q^\P_j
    &\text{if GUD\_ALLOW\_PQUOTA} \\
    G_{j,z} (R^{\P:\C}_j - a_{j,z} R^{\P:\C}_z) f^{\exp\graz}_z
    &\text{else}
    \end{cases}
\\
  &\dots\displaybreak[0]
\\
  g^{\POSi} &= \sum_{j,z} G_{j,z} Q^{\Si}_j
\\
  g^{\PIC} &= \sum_{j,z} G_{j,z} R^\PICPOC_z
\end{align*}
where
\begin{align*}
  \reg^{Q\P}_z &= \left[ \frac{Q^{\P\max}_j - Q^{\P}_j}
                              {Q^{\P\max}_j - Q^{\P\min}_j}
                  \right]_0^1
\\
  &\dots
\\
  \reg^{Q\C}_z &= \min\left\{
        1-\reg^{Q\P}_z, 1-\reg^{Q\N}_z, 1-\reg^{\QFe}_z \right\}
  \qquad\text{(only quota elements)}
\end{align*}

\textbf{Monod-style grazing} (\verb|#define GUD_GRAZING_MONOD_STYLE|)
assumes fixed plankton stoichiometry (i.e., no quotas except possibly Chl), no
grazing inhibition ($i_{\graz}\to\infty$) and a prey-dependent grazing export
fraction, $f^{\exp\graz}_j$ instead of a predator-dependent one.  With these
changes, the above equations apply.



\subsection{Temperature dependence}

\paragraph{GUD\_TEMP\_VERSION 1}
\begin{gather*}
  f^\phy_j(T) = \left[ c_j \left[ e_{1 j}^T
      \ee^{\ds -e_{2 j} {| T - T^{\|opt|}_j |}^{p_j}}  % only with GUD_TEMP_RANGE
    - n \right]_{\ge\num{E-10}}
    \right]^{\le 1}
\\
  f^{\up}_j(T) =
  f^\graz_z(T) =
  f^\remin(T) =
  f^{\|mort|}(T) =
  f^{\|mort2|}(T) =
  1
\end{gather*}
where the exponential is only present with \verb|GUD_TEMP_RANGE|.

\paragraph{GUD\_TEMP\_VERSION 2}
\begin{gather*}
  f^\phy_j(T) = c^{\|Arr|}_j \left[
      \ee^{\ds A^{\|Arr|}_{\|e|}
           \bigl( (T+273.15)^{-1} - {T^{\|Arr|}_{\|ref|}}^{-1} \bigr)}
      \ee^{\ds -e_{2 j} {| T - T^{\|opt|}_j |}^{p_j}}  % only with GUD_TEMP_RANGE
    \right]_{\ge\num{E-10}}
\\
  f^{\up}_j(T) =
  f^\graz_z(T) =
  f^\remin(T) =
  f^{\|mort|}(T) =
  f^{\|mort2|}(T) =
  f_{\|Arr|}(T)
\\
  f_{\|Arr|}(T) = c^{\|Arr|}_j \left[
      \ee^{\ds A^{\|Arr|}_{\|e|}
           \bigl( (T+273.15)^{-1} - {T^{\|Arr|}_{\|ref|}}^{-1} \bigr)}
    \right]_{\ge\num{E-10}}
\end{gather*}
where the second exponential in $f^\phy$ again is only present with \verb|GUD_TEMP_RANGE|.

\paragraph{GUD\_TEMP\_VERSION 3}
\begin{gather*}
  f^\phy_j(T) =
  f^{\up}_j(T) =
  f^\graz_z(T) =
  f^\remin(T) =
  f^{\|mort|}(T) =
  f^{\|mort2|}(T) =
  f^{\|Arr|}(T) =
  \left[ \ee^{\ds A_{\|e|} (T - T_{\|ref|})} \right]_{\ge\num{E-10}}
\end{gather*}

\begin{tabular}{lll}
trait                 & \multicolumn{1}{l}{value}
                                         & \multicolumn{1}{l}{parameter} \\\hline
\noalign{for version 1:\vspace*{1ex}}
\ll{phytoTempCoeff}   & $c_j=1/3$        & \ll{grp\_tempcoeff1}  \\
\ll{phytoTempExp1}    & $e_{1j}=1.04$    & \ll{grp\_tempcoeff3}\quad
                ($E_{\|a|}=\SI{28.023}{\kilo\J\per\mol}$ at $\SI{20}{\degreeCelsius}$) \\
                      & $n=0.3$          & \ll{tempnorm}  \\
\noalign{for version 2:}
                      & $c^{\|Arr|}=0.5882$       & \ll{TempCoeffArr}  \\
                      & $A^{\|Arr|}_{\|e|}=-4000$ & \ll{TempAeArr}\quad
                                           ($E_{\|a|}=\SI{33.257}{\kilo\J\per\mol}$) \\
                      & $T^{\|Arr|}_{\|ref|}=293.15$ & \ll{TempRefArr}  \\
\noalign{for version 3:}
                      & $A_{\|e|}=0.05$   &
                ($E_{\|a|}=\SI{35.725}{\kilo\J\per\mol}$ at $\SI{20}{\degreeCelsius}$) \\
                      & $T_{\|ref|}=20$ \\
\noalign{for \ll{GUD\_TEMP\_RANGE}:\vspace*{1ex}}
\ll{phytoTempExp2}    & $e_{2j}=0.001$   & \ll{grp\_tempcoeff2}  \\
\ll{phytoTempOptimum} & $T^{\|opt|}_j=2$ & \ll{grp\_tempopt}  \\
\ll{phytoDecayPower}  & $p_{j}=4$        & \ll{grp\_tempdecay}  \\\hline
\end{tabular}

(With random generation, $T^{\|opt|}_j$ is drawn from a range
tempmax${}-{}$temprange${}\cdot[0,1]$.)

Version 3 is the same as version 1 with $e_1=1.0512710963760241$,
$c=0.36787944117144233$, $n=0$ and no \verb|GUD_TEMP_RANGE| (except for the constraint
to be smaller than 1).


\subsection{Regularization}

The following are needed so one can turn off P, Si and Fe limitations by setting
half saturations to 0 (and $R^{x:\C}$ resp.\ $V^{x\max}$ also),
\begin{align*}
  \PO_4  &= \max(\eps, \|pTracer|(\|iPO4|))
\\
  \SiO_2 &= \max(\eps, \|pTracer|(\|iSiO2|))
\\
  \FeT   &= \max(\eps, \|pTracer|(\|iFeT|))
\end{align*}
where $\eps=\num{E-38}$.

The following are included so that zero/negative abundances do not give crazy quota
values,
%\[ \X_j = \max(0, c_j) \]
\begin{align*}
  Q^{\P}_j  &= \max(\eps\cdot R^{\P:\C}_j, p_j) / \max(\eps, \X_j) \\
  Q^{\N}_j  &= \max(\eps\cdot R^{\N:\C}_j, n_j) / \max(\eps, \X_j) \\
  Q^{\Si}_j &= \max(\eps\cdot R^{\Si:\C}_j, \si_j) / \max(\eps, \X_j) \\
  Q^{\Fe}_j &= \max(\eps\cdot R^{\Fe:\C}_j, \fe_j) / \max(\eps, \X_j) \\
  \chlc_j &= \max(\eps\cdot R^{\chlc}_j, \chl_j) / \max(\eps, c_j)
\end{align*}
(should use $Q^{\P\min}_j$ instead of $R^{\P:\C}_j$ when running with quota?)
For grazing, quotas are instead set to zero when $\X_j=0$.  (Could use this
also for uptake, but set to $Q^{\P\min}_j$ instead.)

Eqs~(\ref{eq:UNH4})--(\ref{eq:UNO3}) also contain similar regularizations.

None of these should matter for normal abundances/concentrations.



\section{Compiling and Running}

\subsection{Compiling}

Include the word \verb|gud| in \verb|packages.conf| in your code directory.
This will automatically turn on gchem, ptracers and exf.

Set compile-time options for gud in file \verb|GUD_OPTIONS.h| (see table below).

Adjust the number of plankton types, functional groups, autotrophs, grazers,
prey, wavebands and optical types in \verb|GUD_SIZE.h|.

You will also have to adjust the number of passive tracers in
\verb|PTRACERS_SIZE.h|.  You can run \path|MITgcm/mkgudtracers| in your code
directory (after adjusting \verb|GUD_OPTIONS.h| and \verb|GUD_SIZE.h|) to find
out how many ptracers you will need and what they are.

In \verb|GCHEM_OPTIONS.h| you need to define \verb|GCHEM_SEPARATE_FORCING|.

In \verb|EXF_OPTIONS.h| you may want to undef \verb|ALLOW_CLIMSST_RELAXATION|
and \verb|ALLOW_CLIMSSS_RELAXATION| and, if you want to read in sea ice cover,
define \verb|EXF_SEAICE_FRACTION|.

{\renewcommand{\arraystretch}{1.2}
\begin{longtable}[l]{ll>{$}l<{$}l>{$}r<{$}}
\textbf{CPP option} & \textbf{description} \\
\hline
\endhead
  \verb|GUD_ALLOW_NQUOTA|             & enable nitrogen quotas for all plankton \\
  \verb|GUD_ALLOW_PQUOTA|             & enable phosphorus quotas for all plankton \\
  \verb|GUD_ALLOW_FEQUOTA|            & enable iron quotas for all plankton \\
  \verb|GUD_ALLOW_SIQUOTA|            & enable silica quotas for all plankton \\
  \verb|GUD_ALLOW_CHLQUOTA|           & enable chlorophyll quotas for all \emph{phototrophs} \\
  \verb|GUD_ALLOW_CDOM|               & enable a dynamic CDOM tracer \\
  \verb|GUD_ALLOW_CARBON|             & enable air-sea carbon exchange and Alk and O$_2$ tracers \\
  \hline
  \verb|GUD_ALLOW_DENIT|              & enable denitrification code \\
  \verb|GUD_ALLOW_EXUDE|              & enable exudation of individual quotas \\
  \verb|ALLOW_OLD_VIRTUALFLUX|        & enable old virtualflux code for DIC and Alk \\
  \hline
  \verb|GUD_READ_PAR|                 & read PAR from file (unless \verb|GUD_ALLOW_RADTRANS|) \\
  \verb|GUD_USE_QSW|                  & use model shortwave rad. for PAR (unless \verb|GUD_ALLOW_RADTRANS|) \\
  \verb|GUD_AVPAR|                    & compute average PAR in layer, assuming exponential decay (\%) \\
  \verb|GUD_ALLOW_GEIDER|             & enable GEIDER light code \\
  \verb|GUD_ALLOW_RADTRANS|           & enable spectral radiative transfer code (requires GEIDER) \\
  \verb|GUD_CHL_INIT_LEGACY|          & initialize chlorophyll as in darwin2 \\
  \verb|GUD_GEIDER_RHO_SYNTH|         & use $\rho$ instead of acclimated Chl:C for chlorophyll synthesis \\
  \hline
  \verb|GUD_GRAZING_SWITCH|           & enable quadratic grazing as in darwin2+quota \\
  \verb|GUD_ALLOMETRIC_PALAT|         & compute palatability from size ratios \\
  \verb|GUD_NOZOOTEMP|                & turn off grazing temperature dependence \\
  \verb|GUD_TIME_GRAZING|             & split growth and grazing loops so they can be timed separately \\
  \hline
  \verb|GUD_NOTEMP|                   & turn off all temperature dependence \\
  \verb|GUD_TEMP_VERSION|             & select temperature version: 1, 2 or 3 \\
  \verb|GUD_TEMP_RANGE|               & restrict phytoplankton growth to a temperature range \\
  \hline
  \verb|GUD_MINFE|                    & restrict maximum free iron (sic) \\
  \verb|GUD_PART_SCAV|                & enable particle scavenging code \\
  \verb|GUD_IRON_SED_SOURCE_VARIABLE| & enable variable iron sediment source \\
  \hline
  \verb|GUD_DEBUG|                    & turn on debugging output \\
  \verb|GUD_ALLOW_CONS|               & compute and print global element totals \\
  \verb|GUD_UNUSED|                   & value for unused traits \\
  \hline
  \verb|GUD_RANDOM_TRAITS|            & assign traits based on random numbers as in darwin2 \\
  \verb|GUD_TWO_SPECIES_SETUP|        & set traits for darwin2 2-species setup (requires \verb|GUD_RANDOM_TRAITS|) \\
  \verb|GUD_NINE_SPECIES_SETUP|       & set traits for darwin2 9-species setup (requires \verb|GUD_RANDOM_TRAITS|) \\
  \verb|GUD_ALLOW_DIAZ|               & enable diazotrophy when using \verb|GUD_RANDOM_TRAITS| \\
  \hline
\end{longtable}}



\subsection{Running}

You will need to set \verb|useGUD=.TRUE.| in \verb|data.gchem| (and turn on
gchem, ptracers, exf, etc.\ in \verb|data.pkg|).

Runtime parameters are set in \verb|data.gud| in these namelists:

\begin{tabular}{ll}
  \verb|&GUD_FORCING_PARAMS| & parameters related to forcing and initialization \\
  \verb|&GUD_INTERP_PARAMS| & parameters for interpolation of forcing fields (ifdef \verb|USE_EXF_INTERPOLATION|) \\
  \verb|&GUD_PARAMS| & general parameters (not per-plankton traits) \\
  \verb|&GUD_RADTRANS_PARAMS| & parameters for radiative transer \\
  \verb|&GUD_CDOM_PARAMS| & parameters for dynamic CDOM \\
  \verb|&GUD_RANDOM_PARAMS| & parameters for randomly generated traits (deprecated) \\
  \verb|&GUD_TRAIT_PARAMS| & parameters for trait generation (allometric and functional groups)
\end{tabular}

Set initial values/files for the tracers in \verb|data.ptracers|.  You can
generate a template by running \path|MITgcm/mkgudtracers| in your code directory
(get help with the `-h' option).

\begin{sloppypar}
You may generate a minimal file \verb|data.diagnostics| with all the
gud tracers by running \path|MITgcm/mkdiagnosticsdata| in your input/run
directory.
\end{sloppypar}



\subsubsection{Traits}

Traits are generated from the parameters in \verb|&GUD_TRAIT_PARAMS|
(see next section) but can be overridden in \verb|data.traits|:
{\renewcommand{\arraystretch}{1.2}
\begin{longtable}[l]{lllll}
\textbf{trait} & \textbf{dims} & \textbf{symbol} & \textbf{units} & \textbf{description} \\
\hline
\endhead
\multicolumn{5}{@{}l}{\texttt{\&GUD\_TRAITS}}\\
  $\|hasSi|$              & (p)   & $\hasSi$                &  & 1 for diatoms \\
  $\|hasPIC|$             & (p)   & $\hasPIC$               &  & 0: set $\RSUBPICPOC=0$ \\
  $\|diazo|$              & (p)   & $\diazo$                &  & 1: diazotroph, 0: not \\
  $\|useNH4|$             & (p)   & $\useNHiv$              &  & 1: can use $\NH_4$ \\
  $\|useNO2|$             & (p)   & $\useNOii$              &  & 1: can use $\NO_2$ \\
  $\|useNO3|$             & (p)   & $\useNOiii$             &  & 1: can use $\NO_3$ \\
  $\|combNO|$             & (p)   & $\combNO$               &  & 1: combine $\NO_{2/3}$ limit.terms \\
  $\|tempMort|$           & (p)   & $\tempMort$             &  & 1: mortality is temp.dependent \\
  $\|tempMort2|$          & (p)   & $\tempMortTWO$          &  & 1: quadr.mort. is temp.dep. \\
  $\|Xmin|$               & (p)   & $\Xmin$                 & $\unit{mmol C m^{-3}}$ & \\
  $\|amminhib|$           & (p)   & $\amminhib$             & $\unit{(mmol N m^{-3})^{-1}}$ & \\
  $\|acclimtimescl|$      & (p)   & $\acclimtimescl$        & $\unit{s^{-1}}$ & \\
  $\|PCmax|$              & (p)   & $\PCmax$                & $\unit{s^{-1}}$ & \\
  $\|mort|$               & (p)   & $\mort$                 & $\unit{s^{-1}}$ & \\
  $\|mort2|$              & (p)   & $\mortTWO$              & $\unit{(mmol C m^{-3})^{-1} s^{-1}}$ & \\
  $\|ExportFrac|$         & (p)   & $\ExportFrac$           & $\unit{1}$ & export fraction for exudation \\
  $\|ExportFracMort|$     & (p)   & $\ExportFracMort$       & $\unit{1}$ & export fraction for linear mortality \\
  $\|ExportFracMort2|$    & (p)   & $\ExportFracMortTWO$    & $\unit{1}$ & export fraction for quadratic mortality \\
  $\|phytoTempCoeff|$     & (p)   & $\phytoTempCoeff$       & $\unit{1}$ & \\
  $\|phytoTempExp1|$      & (p)   & $\phytoTempExpONE$      & $\unit{$\ln$(degree)}$ & \\
  $\|phytoTempExp2|$      & (p)   & $\phytoTempExpTWO$      & $\unit{(degree C)^{-\phytoDecayPower}}$ & \\
  $\|phytoTempOptimum|$   & (p)   & $\phytoTempOptimum$     & $\unit{degree C}$ & \\
  $\|phytoDecayPower|$    & (p)   & $\phytoDecayPower$      & $\unit{1}$ & \\
  $\|R\_NC|$              & (p)   & $\RSUBNC$               & $\unit{mmol N (mmol C)^{-1}}$ & \\
  $\|R\_PC|$              & (p)   & $\RSUBPC$               & $\unit{mmol P (mmol C)^{-1}}$ & \\
  $\|R\_SiC|$             & (p)   & $\RSUBSiC$              & $\unit{mmol Si (mmol C)^{-1}}$ & \\
  $\|R\_FeC|$             & (p)   & $\RSUBFeC$              & $\unit{mmol Fe (mmol C)^{-1}}$ & \\
  $\|R\_ChlC|$            & (p)   & $\RSUBChlC$             & $\unit{mg Chl (mmol C)^{-1}}$ & \\
  $\|R\_PICPOC|$          & (p)   & $\RSUBPICPOC$           & $\unit{mmol PIC (mmol POC)^{-1}}$ & \\
  $\|wsink|$              & (p)   & $\wsink$                & $\unit{m s^{-1}}$ & \\
  $\|wswim|$              & (p)   & $\wswim$                & $\unit{m s^{-1}}$ & \\
  $\|ksatNH4|$            & (p)   & $\ksatNHiv$             & $\unit{mmol N m^{-3}}$ & \\
  $\|ksatNO2|$            & (p)   & $\ksatNOii$             & $\unit{mmol N m^{-3}}$ & \\
  $\|ksatNO3|$            & (p)   & $\ksatNOiii$            & $\unit{mmol N m^{-3}}$ & \\
  $\|ksatPO4|$            & (p)   & $\ksatPOiv$             & $\unit{mmol P m^{-3}}$ & \\
  $\|ksatSiO2|$           & (p)   & $\ksatSiOii$            & $\unit{mmol Si m^{-3}}$ & \\
  $\|ksatFeT|$            & (p)   & $\ksatFeT$              & $\unit{mmol Fe m^{-3}}$ & \\
  $\|ksatPAR|$            & (p)   & $\ksatPAR$              & $\unit{(uEin m^{-2} s^{-1})^{-1}}$ & (for undef \verb|GUD_ALLOW_GEIDER|) \\
  $\|kinhPAR|$            & (p)   & $\kinhPAR$              & $\unit{(uEin m^{-2} s^{-1})^{-1}}$ & (for undef \verb|GUD_ALLOW_GEIDER|) \\
  $\|inhibcoef\_geid|$    & (p)   & $\inhibcoefSUBgeid$     & $\unit{1}$ & \\
  $\|mQyield|$            & (p)   & $\mQyield$              & $\unit{mmol C (uEin)^{-1}}$ & \\
  $\|chl2cmax|$           & (p)   & $\chlTWOcmax$           & $\unit{mg Chl (mmol C)^{-1}}$ & \\
  $\|grazemax|$           & (p)   & $\grazemax$             & $\unit{s^{-1}}$ & \\
  $\|kgrazesat|$          & (p)   & $\kgrazesat$            & $\unit{mmol C m^{-3}}$ & \\
  $\|palat|$              & (p,p) & $\palat_{j,z}$          & $\unit{1}$ & \\
  $\|asseff|$             & (p,p) & $\asseff$               & $\unit{1}$ & \\
  $\|ExportFracPreyPred|$ & (p,p) & $\ExportFracPreyPred$   & $\unit{1}$ & \\
  $\|respiration|$        & (p)   & $\respiration$          & $\unit{s^{-1}}$ & \\
  $\|Qnmax|$              & (p)   & $\Qnmax$                & $\unit{mmol N (mmol C)^{-1}}$ & \\
  $\|Qnmin|$              & (p)   & $\Qnmin$                & $\unit{mmol N (mmol C)^{-1}}$ & \\
  $\|Qpmax|$              & (p)   & $\Qpmax$                & $\unit{mmol P (mmol C)^{-1}}$ & \\
  $\|Qpmin|$              & (p)   & $\Qpmin$                & $\unit{mmol P (mmol C)^{-1}}$ & \\
  $\|Qsimax|$             & (p)   & $\Qsimax$               & $\unit{mmol Si (mmol C)^{-1}}$ & \\
  $\|Qsimin|$             & (p)   & $\Qsimin$               & $\unit{mmol Si (mmol C)^{-1}}$ & \\
  $\|Qfemax|$             & (p)   & $\Qfemax$               & $\unit{mmol Fe (mmol C)^{-1}}$ & \\
  $\|Qfemin|$             & (p)   & $\Qfemin$               & $\unit{mmol Fe (mmol C)^{-1}}$ & \\
  $\|Vmax\_NH4|$          & (p)   & $\VmaxSUBNHiv$          & $\unit{mmol N (mmol C)^{-1} s^{-1}}$ & \\
  $\|Vmax\_NO2|$          & (p)   & $\VmaxSUBNOii$          & $\unit{mmol N (mmol C)^{-1} s^{-1}}$ & \\
  $\|Vmax\_NO3|$          & (p)   & $\VmaxSUBNOiii$         & $\unit{mmol N (mmol C)^{-1} s^{-1}}$ & \\
  $\|Vmax\_N|$            & (p)   & $\VmaxSUBN$             & $\unit{mmol N (mmol C)^{-1} s^{-1}}$ & \\
  $\|Vmax\_PO4|$          & (p)   & $\VmaxSUBPOiv$          & $\unit{mmol P (mmol C)^{-1} s^{-1}}$ & \\
  $\|Vmax\_SiO2|$         & (p)   & $\VmaxSUBSiOii$         & $\unit{mmol Si (mmol C)^{-1} s^{-1}}$ & \\
  $\|Vmax\_FeT|$          & (p)   & $\VmaxSUBFeT$           & $\unit{mmol Fe (mmol C)^{-1} s^{-1}}$ & \\
  $\|kexcC|$              & (p)   & $\kexcC$                & $\unit{s^{-1}}$ & \\
  $\|kexcN|$              & (p)   & $\kexcN$                & $\unit{s^{-1}}$ & \\
  $\|kexcP|$              & (p)   & $\kexcP$                & $\unit{s^{-1}}$ & \\
  $\|kexcSi|$             & (p)   & $\kexcSi$               & $\unit{s^{-1}}$ & \\
  $\|kexcFe|$             & (p)   & $\kexcFe$               & $\unit{s^{-1}}$ & \\
\hline
\multicolumn{5}{@{}l}{\texttt{\&GUD\_RADTRANS\_TRAITS}}\\
  $\|aphy\_chl|$          & (p,l) & $\aphySUBchl$           & $\unit{m^{-1} (mg Chl m^{-3})^{-1}}$ & \\
  $\|aphy\_chl\_ps|$      & (p,l) & $\aphySUBchlSUBps$      & $\unit{m^{-1} (mg Chl m^{-3})^{-1}}$ & \\
  $\|bphy\_chl|$          & (p,l) & $\bphySUBchl$           & $\unit{m^{-1} (mg Chl m^{-3})^{-1}}$ & \\
  $\|bbphy\_chl|$         & (p,l) & $\bbphySUBchl$          & $\unit{m^{-1} (mg Chl m^{-3})^{-1}}$ & \\
  \hline
\end{longtable}}
Dimensions: p -- nplank, l -- nlam.



\subsubsection{Allometric trait generation}

Plankton types are organized into functional groups.  \verb|grp_nplank(g)|
sets the number of types in group \verb|g|.  Traits may be set the same for all
types in a group, e.g., \verb|grp_diazo(g)|, or based on allometric scaling
relations,
\[
  \mathrm{trait}_j = a_g \cdot V_j^{b_g}
\]
with per-group scaling coefficients $a_g$ and $b_g$.

The volumes $V_j$ of all types can be set in four ways (in order or
decreasing precedence),
\[
  V_j = \begin{cases}
    \verb|grp_biovol(i,g)| &  \\
    V_{\log}(\verb|grp_biovolind(i,g)|) \\
    V_{\log}(\verb|logvol0ind(g)|+i-1) \\
    V_{0 g} f_g^{i-1}
  \end{cases}
\]
where $i$ is the index of type $j$ within the functional group.
$V_{\log}$ is a series of volumes, evenly spaced in log space and defined by
parameters $B=\verb|logvolbase|$ and $I=\verb|logvolinc|$,
\[
  V_{\log} = 10^B, 10^{B+I}, 10^{B+2I}, \dots
\]
and $V_{0 g}=\verb|biovol0(g)|$ and $f_g=\verb|biovolfac(g)|$.

The scaling coefficients are read from namelist \verb|&gud_trait_params| in
\verb|data.gud|.  The following table
shows the correspondence between traits and trait parameters.  Where $b$ is
not given, it is set to 0, i.e., all types in the group share the same trait
value.
{\renewcommand{\arraystretch}{1.2}
\begin{longtable}[l]{ll>{$}l<{$}l>{$}r<{$}}
\textbf{trait} & {\boldmath$a$} & \textbf{default} & {\boldmath$b$} & \textbf{default} \\
\hline
\endhead
  \verb|hasSi|              & \verb|grp_hasSI|                  & 0         \\
  \verb|hasPIC|             & \verb|grp_hasPIC|                 & 0         \\
  \verb|diazo|              & \verb|grp_diazo|                  & 0         \\
  \verb|useNH4|             & \verb|grp_useNH4|                 & 1         \\
  \verb|useNO2|             & \verb|grp_useNO2|                 & 1         \\
  \verb|useNO3|             & \verb|grp_useNO3|                 & 1         \\
  \verb|combNO|             & \verb|grp_combNO|                 & 1         \\
  \verb|tempMort|           & \verb|grp_tempMort|               & 1         \\
  \verb|tempMort2|          & \verb|grp_tempMort2|              & 1         \\
  \verb|Xmin|               & \verb|grp_Xmin|                   & 0.0       \\
  \verb|amminhib|           & \verb|grp_amminhib|               & 4.6       \\
  \verb|acclimtimescl|      & \verb|grp_acclimtimescl|          & 1/20/\day \\
  \verb|mort|               & \verb|grp_mort|                   & 0.02/\day \\
  \verb|mort2|              & \verb|grp_mort2|                  & 0.0       \\
  \verb|ExportFrac|         & \verb|grp_ExportFrac|             & {\rm undef} \\
  \verb|ExportFracMort|     & \verb|grp_ExportFracMort|         & 0.5       \\
  \verb|ExportFracMort2|    & \verb|grp_ExportFracMort2|        & 0.5       \\
  \verb|phytoTempCoeff|     & \verb|grp_tempcoeff1|             & 1/3       \\
  \verb|phytoTempExp1|      & \verb|grp_tempcoeff3|             & 1.04      \\
  \verb|phytoTempExp2|      & \verb|grp_tempcoeff2|             & 0.001     \\
  \verb|phytoTempOptimum|   & \verb|grp_tempopt|                & 2.0       \\
  \verb|phytoDecayPower|    & \verb|grp_tempdecay|              & 4.0       \\
  \verb|R_NC|               & \verb|grp_R_NC|                   & 16/120    \\
  \verb|R_PC|               & \verb|grp_R_PC|                   & 1/120     \\
  \verb|R_SiC|              & \verb|grp_R_SiC|                  & 0.0       \\
  \verb|R_FeC|              & \verb|grp_R_FeC|                  & 0.001/120 \\
  \verb|R_ChlC|             & \verb|grp_R_ChlC|                 & 16/120    \\
  \verb|R_PICPOC|           & \verb|grp_R_PICPOC|               & 0.8       \\
  \verb|wsink|              & \verb|a_biosink|                  & 0.028/\day       & \verb|b_biosink|                   &  0.39 \\
  \verb|wswim|              & \verb|a_bioswim|                  & 0.0              & \verb|b_bioswim|                   &  0.18 \\
  \verb|PCmax|              & \verb|a_vmax_DIC|                 & 1.0/\day         & \verb|b_vmax_DIC|                  & -0.15 \\
  \verb|vmax_NH4|           & \verb|a_vmax_NH4|                 & 0.26/\day        & \verb|b_vmax_NH4|                  & -0.27 \\
  \verb|vmax_NO2|           & \verb|a_vmax_NO2|                 & 0.51/\day        & \verb|b_vmax_NO2|                  & -0.27 \\
  \verb|vmax_NO3|           & \verb|a_vmax_NO3|                 & 0.51/\day        & \verb|b_vmax_NO3|                  & -0.27 \\
  \verb|vmax_N|             & \verb|a_vmax_N|                   & 1.28/\day        & \verb|b_vmax_N|                    & -0.27 \\
  \verb|vmax_PO4|           & \verb|a_vmax_PO4|                 & 0.077/\day       & \verb|b_vmax_PO4|                  & -0.27 \\
  \verb|vmax_SiO2|          & \verb|a_vmax_SiO2|                & 0.077/\day       & \verb|b_vmax_SiO2|                 & -0.27 \\
  \verb|vmax_FeT|           & \verb|a_vmax_FeT|                 & \num{14E-6}/\day & \verb|b_vmax_FeT|                  & -0.27 \\
  \verb|Qnmin|              & \verb|a_qmin_n|                   & 0.07             & \verb|b_qmin_n|                    & -0.17 \\
  \verb|Qnmax|              & \verb|a_qmax_n|                   & 0.25             & \verb|b_qmax_n|                    & -0.13 \\
  \verb|Qpmin|              & \verb|a_qmin_p|                   & 0.002            & \verb|b_qmin_p|                    &  0.00 \\
  \verb|Qpmax|              & \verb|a_qmax_p|                   & 0.01             & \verb|b_qmax_p|                    &  0.00 \\
  \verb|Qsimin|             & \verb|a_qmin_si|                  & 0.002            & \verb|b_qmin_si|                   &  0.00 \\
  \verb|Qsimax|             & \verb|a_qmax_si|                  & 0.004            & \verb|b_qmax_si|                   &  0.00 \\
  \verb|Qfemin|             & \verb|a_qmin_fe|                  & \num{1.5E-6}     & \verb|b_qmin_fe|                   &  0.00 \\
  \verb|Qfemax|             & \verb|a_qmax_fe|                  & \num{80E-6}      & \verb|b_qmax_fe|                   &  0.00 \\
  \verb|ksatNH4|            & \verb|a_kn_NH4|                   & 0.085            & \verb|b_kn_NH4|                    &  0.27 \\
  \verb|ksatNO2|            & \verb|a_kn_NO2|                   & 0.17             & \verb|b_kn_NO2|                    &  0.27 \\
  \verb|ksatNO3|            & \verb|a_kn_NO3|                   & 0.17             & \verb|b_kn_NO3|                    &  0.27 \\
  \verb|ksatPO4|            & \verb|a_kn_PO4|                   & 0.026            & \verb|b_kn_PO4|                    &  0.27 \\
  \verb|ksatSiO2|           & \verb|a_kn_SiO2|                  & 0.024            & \verb|b_kn_SiO2|                   &  0.27 \\
  \verb|ksatFeT|            & \verb|a_kn_FeT|                   & \num{80E-6}      & \verb|b_kn_FeT|                    &  0.27 \\
  \verb|kexcC|              & \verb|a_kexc_c|                   & 0.0              & \verb|b_kexc_c| \\
  \verb|kexcN|              & \verb|a_kexc_n|                   & 0.0              & \verb|b_kexc_n| \\
  \verb|kexcP|              & \verb|a_kexc_p|                   & 0.0              & \verb|b_kexc_p| \\
  \verb|kexcSi|             & \verb|a_kexc_si|                  & 0.0              & \verb|b_kexc_si| \\
  \verb|kexcFe|             & \verb|a_kexc_fe|                  & 0.0              & \verb|b_kexc_fe| \\
  \verb|respiration|        & \verb|a_respir|/\verb|qcarbon|    & 0.0              & \multicolumn{2}{l}{$\num{12E9}\cdot\texttt{qcarbon}$} \\
  \verb|qcarbon|            & \verb|a_qcarbon|                  & \num{1.8E-11}    & \verb|b_qcarbon|                   &  0.94 \\
  \verb|grazemax|           & \verb|a_graz|                     & 21.9/\day        & \verb|b_graz|                      & -0.16 \\
  \verb|kgrazesat |         & \verb|a_kg|                       & 1.0              & \verb|b_kg|                        &  0.00 \\
  \verb|pp_opt|             & \verb|a_prdpry|                   & 1024.0           & \verb|b_prdpry|                    &  0.00 \\
  \verb|pp_sig|             & \verb|grp_pp_sig|                 & 1.0  \\
  \verb|mQyield|            & \verb|grp_mQyield|                & \num{75E-6}  \\
  \verb|chl2cmax|           & \verb|grp_chl2cmax|               & 0.3  \\
  \verb|inhibcoef_geid|     & \verb|grp_inhibcoef_geid|         & 0.0  \\
  \verb|ksatPAR|            & \verb|grp_ksatPAR|                & 0.012  \\
  \verb|kinhPAR|            & \verb|grp_kinhPAR|                & 0.006  \\
  \verb|asseff|             & \verb|grp_ass_eff|                & 0.7   & (nGroup$\times$nGroup) \\
  \verb|ExportFracPreyPred| & \verb|grp_ExportFracPreyPred|     & 0.5   & (nGroup$\times$nGroup) \\
  \hline
  \verb|*phy_chl*| & \multicolumn{4}{l}{assigned according to {\tt grp\_aptype}} \\
  \hline
\end{longtable}}
Palatabilities are initialized to zero and have to be set in \verb|data.traits|
unless \verb|GUD_ALLOMETRIC_PALAT| is defined in which case they are computed
based on predator and prey sizes,
\[
  p_{j,z} = \frac{1}{2\|pp\_sig|_z} \exp\left\{ -\ln(r/\|pp\_opt|_z)^2/(2 \|pp\_sig|_z^2) \right\}
\]
where
\[
  r = V_z/V_j
\]
and \verb|pp_opt| and \verb|pp_sig| are from the above table.

If \verb|GUD_effective_ksat=T|, half saturations for non-quota elements are
computed from quota traits.  The half saturation for $\NO_3$ is computed as
\[
  \ksatNOiii {}' = \frac{ \ksatNOiii \PCmax \Qnmin (\Qnmax-\Qnmin)}
                        { \VmaxSUBNOiii \Qnmax + \PCmax \Qnmin (\Qnmax-\Qnmin)}
\]
and those of the other elements are computed by scaling $\ksatNOiii$ with
the type's elemental ratios.  Here, $\ksatNOiii$ on the right-hand side
is computed from \verb|a_kn_NO3| and \verb|b_kn_NO3|.



\bibliography{mendeley}{}
\bibliographystyle{apalike}

\dots

\end{document}

